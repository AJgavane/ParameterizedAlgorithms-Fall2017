\documentclass{article}
\usepackage{algorithm, algpseudocode}
\usepackage{geometry}
\usepackage{amsmath, amssymb, amsthm, enumerate, hyperref}
\usepackage{color}
\usepackage{setspace}
\usepackage{fancyhdr,lastpage}
\usepackage{url}
\usepackage{tabularx}
\pagestyle{fancy}
\lhead{\footnotesize Opportunity Identification Project}
\chead{}
\rhead{\footnotesize CSC (791/495)-011 -- Fall 2017}
\lfoot{}
\cfoot{\small \thepage/\pageref*{LastPage}}
\rfoot{}


\newcommand{\defproblem}[4]{%
  \hfill\\\smallskip\noindent%
  \begin{tabularx}{\textwidth}{|l X|}%
    \hline%
    \multicolumn{2}{|l|}{\pname{#1}}\\%
    \textbf{Input:}&#2\\%
    \textbf{Parameter:}&#3\\%
    \textbf{Question:}&#4\smallskip\\\hline%
  \end{tabularx}%
  \smallskip%
}%

\newcommand{\pname}[1]{\textnormal{\textsc{#1}}}

\newtheorem*{theorem}{Theorem}
\newtheorem{definition}{Definition}
\newtheorem*{lemma}{Lemma}


\begin{document}

\subsection*{Project Description and Objectives}

\textbf{This project is designed to give you experience in (1) literature review,
(2) where to look for open problems, (3) pitching a research direction (e.g. to
reviewers or funders), and (4) the art of posters.}

\vspace{0.25in}

In this project, teams of two will identify and describe an open problem
in parameterized complexity and/or algorithm design
\emph{which you believe might make a good research project} (e.g.
in this class or for an independent study with Dr. Sullivan's lab)
\footnote{this means that famous open problems are probably not an appropriate choice
unless you have a compelling argument for why some special case is tractable}.

\vspace{0.25in}

You will conduct a small literature review to understand what is known about your
proposed problem (and/or closely related questions), and whether the community has
made any relevant conjectures. It may also be helpful to skim some recent techniques
in the field to see if any seem particularly relevant as potential approaches to solve
the problem. You will then write a brief (2-3 page) report\footnote{The reports will be compiled into a single PDF and made available to the entire class prior to the poster
session to enable more productive discussion and conversation.} and ``pitch'' your problem
to the rest of the class in a mini poster session. \emph{Since some of the proposed problems
will become topics for research projects later in the semester, there's an
incentive to make yours sound interesting and approachable.}

\vspace{0.25in}

The main questions that each of your project artifacts should answer are:
(a) What is the open problem?; (b) Why is it interesting?; (c)
What existing work is closely related?; and (d) Why should someone spend time (or money) on this?.
Note that the answers to some of these may overlap and/or be difficult to detangle,
but all four are important objectives for your audience.

\subsection*{Technical Notes}

Please use the \texttt{\#opportunityID} channel in Slack to discuss general questions
about this project; latex questions (including those about compiling a poster) should go
in \texttt{\#latex}. We will be providing some tips on good sources for open problems,
suggestions for impactful poster design, hints on creating figures, etc in the
coming days, so please watch the channel carefully.

\vspace{0.25in}

You should create a new (private) github repository \texttt{teamX\_project1\_PAC\_2017}
for this project, where \texttt{X}
is the single digit number corresponding to your team assignment on Slack.
Either team member may host the repo, but both members and the instructors
should have read/write access.
\textbf{You must send the repo link to the TA via Slack by Friday, Sept 22.}

\newpage
\subsection*{Report Requirements}

Your report should be between two and three pages plus a bibliography, using
the standard LaTeX article template (and default margins and fonts). You must include:
\begin{enumerate}
  \item A clear, concise abstract
  \item An introduction that motivates and introduces the problem you're proposing.
  Please use a problem box to clearly define the open problem itself.
  \item A short background, including definitions of any non-standard terminology
  \footnote{\emph{please ask on Slack if you have questions about what needs to be included here - many
  common graph theory and parameterized algorithms terms may be assumed}}
  \item A synopsis of related work that describes the history of the problem, whether
  there have been conjectures about it, and/or any work on related problems or special
  cases that you feel is relevant
  \item A discussion of why you feel this is an interesting/approachable open problem
  (e.g. you may choose to describe promising approaches and/or special cases that seem
  tractable).
  \item A bibliography providing complete and corrected formatted citations
\end{enumerate}


\subsubsection*{Submission notes}
\begin{itemize}
\item The report for this project is due at \textbf{9:00 am on Friday, October 6}.
\item Keep all report/poster files in the \texttt{report} folder of the git repository.
\item Name your primary source files \texttt{openproblem\_report.tex}.
\item Upload compiled version named \texttt{compiled\_report.pdf}.

\end{itemize}

\newpage

\subsection*{Poster Requirements}

\begin{itemize}
    \item Your poster should be visually appealing and include
    vector graphics that aid in comprehension and support any
    discussion you envision being common
    \item Although your poster should not be comprehensive (too much
    text is a detraction), it should allow a viewer to understand the
    statement of your problem and what
    the open questions are \emph{without any verbal explanation}
    \item Poster dimensions will be prescribed (and updated here) when
    a template is added to the git repository
    \item \emph{You are responsible for printing your poster}; the CSC department
    has a poster printer available in the mailroom on the third floor of EBII
    (see Carol Allen for access). Do \emph{not} leave printing until the day before!
\end{itemize}

\subsubsection*{Submission notes}
\begin{itemize}
\item Posters will be presented \textbf{in class on Friday, October 13} in two short sessions.
\item You must post in \#general by noon on Thursday, October 12 designating which of
your team will present in each session.
\item Files should be kept in the \texttt{poster} folder of the git repository.
\item Name your primary source file \texttt{openproblem\_poster.tex}.
If you choose to use Powerpoint or OpenOffice to create your poster, you should
instead upload an original source file named
\texttt{openproblem\_poster.}$\{$\texttt{ppt/pptx/odt}$\}$.
\item Upload a compiled version named \texttt{compiled\_poster.pdf} to git.
\end{itemize}


\end{document}
