\documentclass{article}
\usepackage{algorithm, algpseudocode}
\usepackage{geometry}
\usepackage{amsmath, amssymb, amsthm, enumerate, hyperref}
\usepackage{color}
\usepackage{setspace}
\usepackage{fancyhdr,lastpage}
\usepackage{url}
\usepackage{tabularx}
\pagestyle{fancy}
\lhead{\footnotesize Final Project}
\chead{}
\rhead{\footnotesize CSC (791/495)-011 -- Fall 2017}
\lfoot{}
\cfoot{\small \thepage/\pageref*{LastPage}}
\rfoot{}


\newcommand{\defproblem}[4]{%
  \hfill\\\smallskip\noindent%
  \begin{tabularx}{\textwidth}{|l X|}%
    \hline%
    \multicolumn{2}{|l|}{\pname{#1}}\\%
    \textbf{Input:}&#2\\%
    \textbf{Parameter:}&#3\\%
    \textbf{Question:}&#4\smallskip\\\hline%
  \end{tabularx}%
  \smallskip%
}%

\newcommand{\pname}[1]{\textnormal{\textsc{#1}}}

\newtheorem*{theorem}{Theorem}
\newtheorem{definition}{Definition}
\newtheorem*{lemma}{Lemma}


\begin{document}

\section*{Initial Setup Notes}
Please read this document and start discussing the details with your team on \textbf{Friday, October 27}. The TA will make an initial check on these items on \textbf{Tuesday, October 31}, and all items need to be verified by the TA by \textbf{Friday, November 3 at 9:00AM}.

\subsection*{GitHub Repository}

Have someone in your team create a private GitHub repository with the name \texttt{teamX\_project2\_PAC\_2017}, where \texttt{teamX} should be replaced with \texttt{teamA}, \texttt{teamB}, \texttt{teamC}, or \texttt{teamD}. \emph{Do not use the team name from below}. Invite both Dr. Sullivan and Timothy to the repository with read/write access.

\subsection*{Team Name}
Decide on an animal-related team name with some association to your approach or problem, run them by Dr. Sullivan, and then post them in \#general. For example, \texttt{Team Woodpecker} might be a team trying to poke holes into existing claims or approaches, or \texttt{Team KAT} may be using Kernelization and Treewidth as their primary tools. Be creative!\\

While this assignment may seem silly, there is an art to picking a project name that correctly describes your work while also making it pop! In the real world, grant applications may require a name or acronym for your proposal, and picking a name that sticks in reviewers'  minds may determine whether you get the grant or not. Additionally, a smart choice may provide a theme for images, which will help you create engaging posters and talks.

\subsection*{Slack Usage}

We have created a channel for each team; these channels will be renamed once team names have been decided. These channels should  contain anything specific to your project. General assignment questions should be posted on \texttt{\#general} channel, and latex questions (including those about the report or talk template) should go in \texttt{\#latex}.

\subsection*{Final Report}
The final report should be in LIPIcs style, which is a popular \LaTeX ~template for conference in theoretical computer science. You can find this template online, and we have also added it to the templates folder here: \url{https://github.com/bdsullivan/ParameterizedAlgorithms-Fall2017/tree/master/templates/final_report}.

The requirements for the final report are as follows:
\begin{itemize}
\item Written in \LaTeX with the LIPIcs style.
\item 10 pages, including references.
\item An optional 2 page appendix for additional proofs, tables, and figures. \textbf{Note}: \emph{The appendix is read at the discretion of the reviewer and should not contain major results}.
\end{itemize}

\newpage
Please initialize your final report with the following:
\begin{itemize}
\item Add a \texttt{report} folder to your team's repository, keep all items relevant to the final report in this folder.
\item Add a \texttt{report.tex} file that uses the LIPIcs style file.
\item Initialize the report by adding a title, your team name, and the name of each member.
\item Provide a \texttt{Makefile} that compiles your project using \texttt{pdflatex} and \texttt{bibtex}.
\end{itemize}

\subsubsection*{Submission notes}
\begin{itemize}
\item These items will be overviewed in class on \textbf{Friday, October 27}, and the TA will be available for questions.
\item Try to get all items finished by \textbf{Tuesday, October 31}, at this point the TA will check in with every team. Any issues must be resolved by \textbf{Friday, November 3 at 9:00AM}.
\end{itemize}

\end{document}
