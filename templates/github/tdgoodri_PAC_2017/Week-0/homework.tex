\documentclass{article}
\usepackage{algorithm, algpseudocode}
\usepackage{amsmath, amssymb, amsthm}
\usepackage{color}
\usepackage{enumerate}
\usepackage{geometry}
\usepackage{hyperref}

\newtheorem{theorem}{(False) Theorem}
\newtheorem{definition}{Definition}

\title{CSC791 Week 0: Homework 0}
\author{Timothy Goodrich}

\begin{document}
\maketitle

\subsection*{Problem 1}
\begin{theorem}
All students at NC State have the same major.
\end{theorem}

\begin{proof}
We proceed by induction:

\noindent\underline{Base case:} $n = 1$. One NC State student has the same major, so the claim is true.

\noindent\underline{Induction step:} Suppose the claim is true for $k$ students, we will show that it is also true for $k+1$ students. Label our students $S = \{x_1, x_2, \dots, x_{k+1}\}$ and split the students into two subsets: $S_1 = S \setminus \{x_2\}$ and $S_2 = S \setminus \{x_3\}$. By the induction hypothesis, all students in $S_1$ have the same major, and all students in $S_2$ have the same major. But $x_1$ is in both sets, therefore all students have the same major.
\end{proof}

\end{document}
